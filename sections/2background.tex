\section{Background}
\label{background}

% -----------------------------------------------------

The underlying organizational constructs that this protocol utilizes is based upon
those defined in \cite{wood2014ethereum} \S2. This organizational model defines
that are reused in our protocol, these constructs are the \textit{transaction} and
the \textit{block}. For the purposes of this paper, we view a transaction as
unique tokens (ex. a unique 32-byte hash), and a block as a data structure that
bundles an ordered series of transactions. As with earlier blockchain protocols,
blocks in this scheme must reference a valid, previously published unique block
(commonly referred to as a parent).

The purpose of Proof of Work is to provide an easily verifiable cryptographic proof
that a certain amount of work has been done. It is used by the protocol to generate
the notion of \textit{difficulty}, in that proofs generated from a block with higher
difficulty require more computational resources than that of a lower difficulty.
Therefore, difficulty is a measure of total \textit{work} done to produce a block.
This numerical value in turn can be utilized to determine the link of blocks that
represent the chain with the \textit{most work} via summation, among other approaches
(this paper included) in agreement on a canonical chain. This in turn enforces agreement
on a canonical ordering of transactions, which forms the basis for all blockchain interactions.



% -----------------------------------------------------