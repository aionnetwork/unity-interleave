\section{Background}
\label{background}

Assumed a simplified model for a blockchain consisting of \textit{blocks} and \textit{transactions}. Assume each transaction is some arbitrary unique token (e.g. a unique 32-byte hash), and each block is a data structure that
bundles an ordered series of transactions. As with earlier blockchain protocols, blocks in this scheme must reference a valid, previously published unique block (commonly referred to as a parent). See \cite{wood2014ethereum} for a more complete model for a blockchain. 

The purpose of Proof of Work (\cite{nakamoto2008}) is to provide an easily verifiable cryptographic proof that a certain amount of work has been done. It is used by the protocol to generate the notion of \textit{difficulty}, in that proofs generated from a block with higher difficulty require more computational resources than that of a lower difficulty. Therefore, difficulty is a measure of total \textit{work} done to produce a block. This numerical value in turn can be utilized to determine the link of blocks that represent the chain with the \textit{most work} via summation, among other approaches (this paper included) in agreement on a canonical chain. This in turn enforces agreement on a canonical ordering of transactions, which forms the basis for all blockchain interactions.
