\section{Introduction}
\label{intro}

The idea of \emph{Proof of Work (PoW)} originated in the work of Dwork and Naor
in the 1990s\cite{dwork-naor}. Used in the majority of blockchain protocols today, including
Bitcoin, the idea is that a block can only be considered valid if it includes
a solution to some cryptographic puzzle that is hard to solve, but easy to verify. 
Block producers on a PoW network are often called \emph{miners}, and 
the security of a PoW network relies on the difficulty of finding solutions
to these puzzles. For more on PoW consensus, see \cite{nakamoto2008}.

A PoW network is considered to be secure if a large percentage of the hash power
solving these puzzles is controlled by honest actors (usually stated as 51\% of the hash power,
though the number can be as high as 66\% \cite{selfish_mining}). 
The rise of large mining pools \cite{aliaga2018}, as well as the new possibility of renting
hash power makes it easier for a single entity to acquire large quantities of hash power,
especially on less well-secured networks. This revokes previous assumptions around
the security of blockchains, leaving smaller non-ASIC networks especially vulnerable \cite{sinnige2018}.

An alternative approach known as \emph{Proof-of-Stake (PoS)} encompasses a broad spectrum of
consensus protocols that share the common attribute in utilizing the \textit{stake} a user has in a network \cite{BentovGM14}.
Block producers on a PoS network are often called \emph{stakers}, and 
this stake is usually some economic asset that lives on the network itself.
The security of networks built on PoS consensus relies on incentivising block producers
to support the network they are on (since they have assets on the network), and by 
punishing behaviour that can be shown to be dishonest (by confiscating the assets of
a dishonest actor). PoS networks are vulnerable to a different set of attacks than PoW networks,
see \cite{brown2018formal} for more.

In this paper, we present a new consensus protocol that combines the ideas
of PoW and PoS. This protocol, which we call ``Unity-Interleave'', is a modification
of the protocol described in \cite{wu2019unifying}.
The approach is to simply alternate between PoW and PoS blocks; a miner
cannot produce a PoW block on top of another PoW block, and the same is
true of stakers and PoS blocks. We strive to achieve the benefits of both PoW and
PoS protocols, while ameliorating their vulnerabilities.

\textbf{Paper Layout.} In \S\ref{background} we go over some of the background of blockchain technology.
\S\ref{unity} defines the protocol itself in detail, and \S\ref{analysis} 
analyses the protocol's resistance to various attacks. Finally, in \S\ref{results},
we present the results of some simulations that prove our protocol works as desired.