\section{Analysis}
\label{analysis}

In this section, we assess the protocol's resistance to various attacks. We claim that
in most cases, the protocol behaves no worse than a pure PoW network. 

\subsection{Double-spend attack}
\label{double_spend_attack}

The \textit{double-spend attack}
is the well-known, but somewhat poorly named, ``51\% attack'' \cite{selfish_mining}. In the simplest version of this attack, a malicious agent
sends some transaction to the network to reap an off-chain benefit (say she sells a large
amount of her on-chain assets on an exchange), and then releases a side-chain in which
that transaction never occurred. If this side-chain becomes the main-chain, she has 
exploited the network.

Large parts of the protocol were designed with the specific idea of frustrating
this attack. The interleaving of PoW and PoS blocks serves to make this attack much harder to launch.

We model this attack by letting the average difficulty of PoW blocks on the chain
be $d_w$, and the average difficulty of PoS blocks $d_s$.
We let the attacker have $k$ times the hashpower and $l$ times
the stake of the honest block producers on the network. Once the difficulty has adjusted on the
attacker's side-chain, she will be producing PoW blocks of average difficulty $kd_w$, and PoS
blocks of average difficulty $ld_s$. She should expect to win if her chain grows heavier 
than the honest-chain in the long-run. Mathematically, this means she needs
$$kd_w + ld_s > d_w + d_s$$

We can easily observe that if $k < 1$ and $l < 1$, the attacker is not expected to win, and
if \textit {both} $k>1$ and $l>1$, the attacker is always expected to win.
Therefore, 51\% of at least one
domain is necessary to expect to win,
and 51\% of both is sufficient to expect to win. 

The intermediate case is when the attacker has a majority of one domain,
but not the other. We let the attacker have $k>1$ and $l<1$ (the other case
is symmetric). We can rewrite the winning condition as 
$$k > 1 + (1-l)\frac{d_s}{d_w}$$

Note that the difficulty values $d_w$ and $d_s$ are just indications
of how hard it was to produce a block, and therefore can be controlled to some
extent by the designers of a protocol. This gives us
the nice property that by manipulating these difficulty values, 
we can make the attack harder to launch in one direction than the other. On the OAN,
for instance, we believe that it will be easier for attackers to acquire a majority of the
hash power ($k > 1$) than a majority of the stake. Accordingly, we set the typical
$d_s$ to be several orders of magnitude larger than $d_w$ (roughly $10^{14}$ times
larger). The result is that an attacker with, say, 9/10ths the stake of the
honest network still needs to have over $10^{10}$ times as much hash power as the honest
work. The converse effect is that an attacker with a majority of the stake needs
very little hash power to dominate the network, but we believe acquiring a majority
of the stake to be infeasibly difficulty.

\subsection{Nothing-at-Stake}

The  \textit{Nothing-at-Stake} problem arises in pure PoS blockchains when stakers
try to produce blocks on every branch they see \cite{brown2018formal}. Since there
is no cost to PoS block production, this is a profit-maximising strategy for PoS block
forgers. This can lead to a fragile network with many branches and side-chains. 

Our protocol is not susceptible to this problem since the miners will resolve the 
branch; of several valid PoS blocks at the same level, whichever one has a PoW block
built on top of it first will likely be included into the canonical chain.

\subsection{Stake Grinding}

A PoS block producer is said to be \textit{stake grinding} if, in the production of
a block, they can somehow increase the chances that they produce the next block too
\cite{buterin_randomness}. In our protocol, the randomness that determines block production is entirely
determined by the seed of the previous PoS block, and the private key of the block producer
\S\ref{pos_block_generation}.

At every block an attacker will try to brute force compute an account that generates a lower delta
for the next $k$ blocks, then attempt to transfer ownership of coins over to that account. Under our model
the only way to achieve this is through unlocking then locking one's coins under such an account.
Suppose that this would be done instantaneously, then the attacker could wait until $seed_{n-1}$ and
simultaneously bet that he or she can win the next block honestly $seed_n$ \textit{and} compute
the account that generates a lower delta given $seed_n$. However, notice that the attacker must
at least win 1 block for this attack to work. Therefore, if we enforce a lockout period of $x$,
the attacker must now win at least $x+1$ blocks consecutively the probability of
success for such an attack translates to,

\begin{equation}
    \label{eq:attacker_success}
    \mathbb{P}(\texttt{win x + 1 consecutive blocks})=p^{x+1}
\end{equation}

Where $p$ is the probability of success, therefore $p=s_{i}/\sum_i{s_i}$ (the probability of winning
the next block), assuming that $\sum_i{s_i}$ is constant for the next $x$ blocks.

\subsection{Denial of Service}

Since our protocol strictly interleaves PoW and PoS blocks, the network stalls if miners 
or stakers do not produce a block for some reason. This is a concern, but it is not 
significantly worse than a pure PoW network. 

\subsection{Selfish Mining}

The problem of \textit{selfish mining} in pure PoW networks is that miners who
adopt this strategy hold onto blocks, mining from an advantageous position \cite{selfish_mining}.
This is done until such a point that the network catches up, in which case they
will produce their chain. The net effect of this is in deliberately rendering
the work of the honest workers wasted, thus netting the miner a block production
rate disproportionate to their actual mining power. Our protocol does not suffer from this
problem specifically, as mining and staking blocks must be interleaved.

There is, however, a similar problem that our protocol is vulnerable to. A staker can collude 
with a miner by sending the miner a PoS block before the waiting time has elapsed.
This miner now gets a head start on mining a PoW block on top of it. This costs the staker
nothing (since she can still release her block to the wider network when it becomes
eligible). It does give the miner the chance to win more of the blocks than their
share of the hash power.

